
\documentclass[12pt,aspectratio=169,notheorems]{beamer}

% Packages
\usepackage{mathtools}
\usepackage{hyperref}
\usepackage{setspace}
\usepackage{tikz}
\usetikzlibrary{positioning}
\usepackage{ragged2e}
\usepackage{nicematrix}
\usepackage[compatibility=false]{caption}
\usepackage[framemethod=tikz]{mdframed}

% Colored boxes
\usepackage{tcolorbox}
\tcbsetforeverylayer{autoparskip} % Removes unwanted vertical space

% Theme
\usetheme{metropolis}
\metroset{subsectionpage=progressbar,sectionpage=none}
\setbeamertemplate{theorems}[numbered]

\usepackage{fontspec}
\defaultfontfeatures{LetterSpace=30}


% Code
\usepackage{listings}
% Custom colors
\definecolor{codegreen}{rgb}{0,0.6,0}
\definecolor{codegray}{rgb}{0.5,0.5,0.5}
\definecolor{codepurple}{rgb}{0.58,0,0.82}
\definecolor{backcolour}{rgb}{0.95,0.95,0.92}
\definecolor{light-gray}{gray}{0.95}
% Python style for highlighting
\newcommand\pythonstyle{\lstset{
		backgroundcolor=\color{white},
		language=Python,
		basicstyle=\ttfamily\footnotesize,
		morekeywords={self},              % Add keywords here
		keywordstyle=\color{blue},
		stringstyle=\footnotesize\color{deepgreen},
		commentstyle=\color{codegreen},
		showstringspaces=false,
		frame=tb, 
		numbers=left,
		numbersep=5pt,
		numberstyle=\tiny\color{gray},
		columns=fullflexible,
		numbersep=5pt,
		aboveskip=1em,
		showtabs=false,
		tabsize=2,
		gobble=3
}}

% Python environment
\lstnewenvironment{python}[1][]
{
	\pythonstyle
	\lstset{#1}
}
{}
% Python for inline
\newcommand\pythoninline[1]{{\pythonstyle\lstinline!#1!}}

% Tables and justification
\newcommand{\ra}[1]{\renewcommand{\arraystretch}{#1}}
\justifying

% Subfigures
\setbeamertemplate{caption}[numbered]
\usepackage[caption=false,font=footnotesize]{subfig}

% Colors
\setbeamercolor{uppercolgreen}{fg=white,bg=green!35}
\setbeamercolor{lowercolgreen}{fg=black,bg=green!10}
\setbeamercolor{uppercolred}{fg=white,bg=red!35}
\setbeamercolor{lowercolred}{fg=black,bg=red!10}
\setbeamercolor{uppercolblue}{fg=white,bg=blue!35}
\setbeamercolor{lowercolblue}{fg=black,bg=blue!10}
\definecolor{darkred}{rgb}{0.55, 0.0, 0.0}
\definecolor{darkpastelblue}{rgb}{0.47, 0.62, 0.8}
\definecolor{darkcerulean}{rgb}{0.03, 0.27, 0.49}
\definecolor{darkgreen}{rgb}{0.0, 0.55, 0.0}
\definecolor{deepblue}{rgb}{0,0,0.5}
\definecolor{deepred}{rgb}{0.6,0,0}
\definecolor{deepgreen}{rgb}{0,0.5,0}

% Matrix size
\usepackage{stackengine} 
\stackMath
\def\sss{\scriptstyle \color{gray}}
\setstackgap{L}{12pt}
\def\stacktype{L}

% Mathematical commands
\newcommand{\vc}[1]{\boldsymbol{\mathbf{#1}}}
\newcommand{\pr}[1]{{#1\,}'}
\newcommand{\idx}[2]{{\color{darkpastelblue}[}#1{\color{darkpastelblue}]_{#2}}}
\newcommand{\shape}[2]{\stackunder{#1}{\sss #2}}
\DeclareMathOperator*{\argmax}{arg\,max}
\DeclareMathOperator*{\argmin}{arg\,min}

% Style commands
\newcommand{\myalert}[1]{{\color{darkred}\textbf{#1}}}

% White frame (for images)
\newenvironment{whiteframe}[1]{
	\setbeamercolor{background canvas}{bg=white}
	\begin{frame}{#1}
	}{
	\end{frame}
}

\setbeamerfont{frametitle}{size=\normalsize}
\setbeamertemplate{frametitle}[default][right]
{	
}
%\setbeamercolor{frametitle}{fg=white,bg=black!75}

% No headline frame
\makeatletter
\newenvironment{noheadline}{
	\setbeamertemplate{headline}{}
	\addtobeamertemplate{frametitle}{\vspace*{-3.5\baselineskip}}{}
}{}
\makeatother

% Lists
\setbeamertemplate{itemize items}[triangle]

% Footnotes
\setbeamertemplate{footnote}%
{%
	\parindent 1.5em\noindent%
	\justifying\setstretch{0.7}%
	\hbox to 1em{\hfil\insertfootnotemark}\begin{scriptsize}\insertfootnotetext\end{scriptsize}\par%
}

% Footnote without number
\newcommand\blfootnote[1]{%
	\begingroup
	\renewcommand\thefootnote{}\footnote[frame]{\noindent#1}%
	\addtocounter{footnote}{-1}%
	\endgroup
}

% Theorems
\newcommand*{\theorembreak}{\usebeamertemplate{theorem end}\framebreak\usebeamertemplate{theorem begin}}
\newcounter{theo}[section]\setcounter{theo}{0}
\renewcommand{\thetheo}{\arabic{section}.\arabic{theo}}

\newenvironment{theo}[2][]{%
	\refstepcounter{theo}%
	\ifstrempty{#1}%
	{\mdfsetup{%
			frametitle={%
				\tikz[baseline=(current bounding box.east),outer sep=0pt]
				\node[anchor=east,rectangle,fill=red!10]
				{\strut Theorem~\thetheo};}}
	}%
	{\mdfsetup{%
			frametitle={%
				\tikz[baseline=(current bounding box.east),outer sep=0pt]
				\node[anchor=east,rectangle,fill=red!10]
				{\strut Theorem~\thetheo:~#1};}}%
	}%
	\mdfsetup{innertopmargin=5pt,linecolor=red!10,%
		linewidth=2pt,topline=true,%
		frametitleaboveskip=\dimexpr-\ht\strutbox\relax
	}
	\begin{mdframed}[]\relax%
		\label{#2}}{\end{mdframed}}


\setlength{\arrayrulewidth}{0.5mm}
\setlength{\tabcolsep}{10pt}
\renewcommand{\arraystretch}{1.3}

\begin{document}

{\usebackgroundtemplate{\tikz\node[opacity=0.4,inner sep=0]{\includegraphics[width=\paperwidth,height=\paperheight]{header}};}%
\begin{frame}[plain]
	\vspace{0.5cm}
	\title{\large \begin{spacing}{1.0}Fondamenti di Machine Learning\end{spacing}\vspace{0.25em}
		\normalsize \begin{spacing}{1.0}\textbf{Laurea Triennale in Ingegneria delle Comunicazioni}\end{spacing}\vspace{0.5em}}
	\subtitle{\Large \textbf{1 - Introduzione}}
	\date{
		{\includegraphics[scale=0.8]{Uniroma1}}
	}
	\author{
		\setlength{\tabcolsep}{2pt}
		\begin{tabular}{rl}
			\textbf{Docente}: & S. Scardapane \\
		\end{tabular}
	}\titlepage
\end{frame}
}

\part{1}

\section{Introduzione}
\subsection{Che cosa è il machine learning?}


\begin{frame}{Il diavolo è nei dettagli}

	Un neolaureato in una filiale di una banca si vede assegnati due task da risolvere:

	\vspace{0.5em}
	{\large\myalert{Task 1}}
	\vspace{5pt}
	{\color{gray}\hrule}
	Filtrare un insieme di {\color{blue!80}\textbf{email}} in base alla {\color{blue!80}\textbf{dimensione dei loro allegati}}.

	\vspace{1em}
	{\large\myalert{Task 2}}
	\vspace{5pt}
	{\color{gray}\hrule}
	Filtrare un insieme di {\color{blue!80}\textbf{clienti}} in base alla {\color{blue!80}\textbf{probabilità che vadano in default}}.

	\vspace{0.5em}
	Il primo task può essere risolto facilmente, mentre il secondo è un buon esempio di \textbf{apprendimento automatico}.

\end{frame}

\begin{frame}{Il beneficio del senno di poi}

	Il concetto di \textit{probabilità di default} non è facile da definire, ma diventa possibile avendo uno storico: ad esempio, se i clienti sono andati in negativo in passato, sono (probabilmente) \textit{inattendibili}.

	\textbf{L'apprendimento automatico} (ML), ed in particolare \textbf{l'apprendimento supervisionato}, permettono di sfruttare questo concetto \textit{inferendo} e \textit{generalizzando} la relazione da dati storici a \textit{nuovi potenziali clienti}.

	\begin{tcolorbox}
		Questo è solo un esempio didattico, ed il \myalert{credit scoring} tramite ML è una pessima idea per varie ragioni, come vedremo più avanti.
	\end{tcolorbox}

\end{frame}

\begin{frame}{Lo storico (il dataset)}

	Un \myalert{dataset} è un insieme di coppie (input, output) che descrivono \myalert{esempi} di quanto vorremmo ottenere (es., stimare il credit score).

		{\footnotesize
			\begin{center}
				\begin{tabular}{ |F{1.5cm}|F{2.8cm}|F{2.3cm}|F{1.8cm}|T{2cm}|  }
					\hline
					Età & Cliente da (anni) & Disponibilità & Prestito & \textbf{Default?} \\
					\hline
					18  & 2                 & 3000          & 1000     & {0}               \\
					42  & 5                 & 10000         & 2000     & {1}               \\
					... & ...               & ...           & ...      & ...               \\
					\hline
				\end{tabular}
			\end{center}
		}

		Date le prime 4 colonne (\myalert{input}), vogliamo predire la colonna finale (\myalert{output}). Ogni colonna di input viene detta una \myalert{feature}, mentre l'ultima colonna viene detta \myalert{target}. Il dataset viene detto \myalert{tabellare}.

\end{frame}

\begin{frame}{Tipi di feature}

	\begin{itemize}
		\item In questo corso consideriamo due tipi di feature molto semplici: feature \myalert{numeriche} (numeri reali), e feature \myalert{categoriche} (valori discreti).
		\item Il target, invece, può essere a sua volta un valore continuo (es., prezzo di una casa) o discreto (es., classe di un cliente). Parliamo rispettivamente di \myalert{regressione} e \myalert{classificazione}.
		\item Le feature possono essere direttamente in nostro possesso, oppure estratte in qualche modo dai dati (\myalert{feature engineering}).
	\end{itemize}

\end{frame}

\begin{frame}{Modelli ed algoritmi di apprendimento}

	Un \myalert{modello} è una funzione che prende in input le feature e produce un output desiderato (es., la classe): 

	\begin{equation}
		\hat{y} = f(x)
	\end{equation}

	Un \myalert{algoritmo di apprendimento} è un algoritmo che, dato un dataset, produce una istanza particolare della classe di modelli scelta: diciamo che stiamo \myalert{addestrando} il modello. 
	
	Esistono numerosissimi modelli e algoritmi di apprendimento per dataset tabellari, ciascuno con le sue proprietà e limiti, che vedremo più avanti.

\end{frame}

\begin{frame}{Modelli ed apprendimento}


	\begin{itemize}
		\item Il modello descrive in che modo \textit{rappresentare} la relazione tra input ed output: alberi decisionali, regole logiche, reti neurali...
		\item L'algoritmo di apprendimento descrive come costruire un modello che abbia buone prestazioni sul proprio dataset.
	\end{itemize}

	Come vedremo, scegliere questi due aspetti dipende da tanti fattori: capacità di rappresentazione, tipo di dato, efficienza computazionale. Dobbiamo anche avere buone \myalert{metriche} per valutare le performance.

\end{frame}

\begin{whiteframe}{Visione d'insieme}

	\begin{figure}
		\includegraphics[width=0.8\pagewidth]{ML.pdf}
		\caption{Componenti essenziali di un sistema di apprendimento automatico: definizione delle \myalert{feature}, \myalert{dataset}, \myalert{modelli}, e \myalert{algoritmi di apprendimento}.}
	\end{figure}

\end{whiteframe}

\begin{frame}{Collezionare un dataset}

	Un \myalert{dataset} è un insieme di coppie (input, output) che descrivono \myalert{esempi} di quanto vorremmo ottenere (es., stimare il credit score). Costruire un buon dataset non è banale ed è solitamente la parte più complessa del problema:

	\begin{itemize}
		\item Deve essere \textit{rappresentativo} di tutti i possibili clienti (passati e futuri).
		\item Deve essere sufficiente \textit{numeroso} per permettere di generalizzare a nuovi clienti.
	\end{itemize}

	Ritorneremo su questi aspetti più avanti quando definiremo il problema in maniera formale.

\end{frame}

\begin{whiteframe}{Esempio di domain shift}

	Spesso i dati su cui vengono allenati i modelli non corrispondono ai dati su cui vengono richieste predizioni: in questo caso, si parla di \myalert{domain shift}.

	\begin{figure}
		\includegraphics[width=0.5\pagewidth]{DomainShift.pdf}
		\caption{Un esempio artificiale di domain shift: una banca con clienti prevalentemente giovani allena un modello per uso interno, ma non può applicarlo su nuovi clienti che non rientrano in queste caratteristiche.}
	\end{figure}

\end{whiteframe}



\begin{frame}{Pre-processing delle feature}

	Questo semplice esempio chiarisce anche che in molti casi è necessaria una fase di \myalert{pre-processing} dei dati per renderli più facilmente manipolabili, es.:
	%
	\begin{itemize}
		\item Colonne diverse hanno diversi range (\myalert{normalization});
		\item Alcuni dati potrebbero essero mancanti (\myalert{missing data imputation});
		\item Alcuni valori potrebbero essere incorretti o sbagliati (\myalert{outlier detection} or \myalert{anomaly detection});
		\item Nuove feature possono essere estratte da feature in nostro possesso, es., il CAP a partire dall'indirizzo (\myalert{feature engineering}).
	\end{itemize}

\end{frame}



\begin{frame}[fragile]{Overfitting e bias}

	Un famoso aneddoto (con buona probabilità apocrifo)\footnote[frame]{https://gwern.net/tank} descrive una rete neurale allenata a riconoscere carri armati. Per errore, tutte le foto di carri armati nel training set sono scattate in giorni nuvolosi, e la rete `impara' questa correlazione. Le prestazioni in fase di training sono eccellenti, mentre in fase di \textit{deployment} sono pessime.

	Questo è un esempio molto artificiale di \myalert{bias} nella collezione dei dati. Lo scarto di prestazione viene detto \myalert{overfitting}.

\end{frame}

\begin{whiteframe}{Husky o neve?}

	\begin{figure}
		\includegraphics[width=0.6\pagewidth]{images/husky}
		\caption{Un famoso esempio (artificiale) di bias (riprodotto da Ribeiro et al., 2016): un sistema allenato a riconoscere gli husky impara in realtà a riconoscere la neve sullo sfondo.}
	\end{figure}

\end{whiteframe}

\begin{whiteframe}{Il caso COMPAS (2016)}

	\begin{figure}
		\includegraphics[width=0.8\pagewidth]{images/propublica}
		\caption{COMPAS è un esempio importante di bias in un sistema usato in ambito giudiziario, a seguito di una indagine di Pro Publica nel 2016. L'inchiesta ha avuto un ampio seguito mediatico ed accademico, in quanto le conclusioni possono variare a seconda di quale metrica di `unfairness' scegliamo di usare.}
	\end{figure}

\end{whiteframe}




\subsection{Machine learning e reti neurali}

\begin{whiteframe}{ChatGPT (dall'alto)}

	Un modello come ChatGPT genera una distribuzione sul \myalert{prossimo blocco di testo} (token), quindi viene chiamato un \myalert{language model}. Usandolo ripetutamente possiamo generare testi molto lunghi (\myalert{autoregressione}).

	\begin{figure}
		\includegraphics[width=0.7\columnwidth]{images/LLM}
	\end{figure}

	Nella terminologia di prima, il prompt è l'\textit{input}, mentre la generazione è l'\textit{output}, e la rete neurale è il \textit{modello}.

\end{whiteframe}


\begin{whiteframe}{StableDiffusion e modelli generativi di immagini}

	\begin{figure}
		\includegraphics[height=0.8\textheight]{images/StableDiffusion}
		\caption{\href{https://www.reddit.com/r/midjourney/comments/129pek3/mcu_characters_as_80s_wrestlers/}{MCU Characters as 80s Wrestlers [Reddit]}}
	\end{figure}

\end{whiteframe}

\begin{whiteframe}{StableDiffusion, sempre dall'alto}

	Un modello generativo come Stable Diffusion rientra sempre nell'ambito dell'apprendimento supervisionato, ma in questo caso l'input è una descrizione testuale e l'output una immagine:
	%
	\begin{figure}
		\includegraphics[width=0.7\columnwidth]{images/Text2Image}
	\end{figure}

\end{whiteframe}

\begin{frame}{Un confronto}

	Stable Diffusion e ChatGPT sono a loro volta esempi di apprendimento automatico, essendo allenati a partire da dataset di testo o di coppie (immagine, descrizione). Hanno però diverse caratteristiche peculiari:
	%
	\begin{itemize}
		\item I loro input ed output possono essere estremamente complessi (testo, immagini, video, ...).
		\item Di conseguenza, i dataset richiesti possono a loro volta essere molto grandi (milioni di immagini, PB di testo).
	\end{itemize}
	%
	Di tutti i modelli di apprendimento supervisionato, le \myalert{reti neurali} sono l'unico metodo noto che possono scalare a questi contesti. In questo corso vedremo solo le basi per quanto riguarda le reti neurali.

\end{frame}

\subsection{Altri argomenti}

\begin{frame}{Oltre l'apprendimento supervisionato}

	L'apprendimento supervisionato è solo un sottoinsieme del campo del machine learning, seppur il più comune. Altre categorie interessanti di apprendimento automatico sono:
	%
	\begin{itemize}
		\item \myalert{Unsupervised learning}: in questo caso il nostro obiettivo non dipende da una etichetta (\textit{label}) associato agli input. Ad esempio nel \myalert{clustering} vogliamo raggruppare i nostri dati in insiemi sufficientemente omogenei. Nella \myalert{dimensionality reduction}, vogliamo proiettarli in uno spazio a 2D o 3D per la visualizzazione.
		\item \myalert{Reinforcement learning} (apprendimento con rinforzo): in questo caso il modello deve prendere diverse decisioni in sequenza, e riceve un reward solo al termine di ogni sequenza (es., giocare a scacchi).
	\end{itemize}

\end{frame}

\begin{whiteframe}{Visualizzazione del clustering}

	Il clustering può essere inteso come un problema di classificazione per il quale non conosciamo le etichette in fase di training, es., segmentare i clienti di una azienda in gruppi non noti a priori per fini di marketing.
	%
	\begin{figure}
		\includegraphics[width=0.5\columnwidth]{images/1200px-Cluster-2.svg.png}
		\caption{Esempio di clustering (riprodotto da Wikipedia).}
	\end{figure}

\end{whiteframe}

\begin{frame}{Data cleaning e preprocessing}

	Esistono poi argomenti ortogonali alla scelta del modello e del tipo di apprendimento, che approfondiremo solo in parte:
	%
	\begin{itemize}
		\item \myalert{Misura delle performance} (valutazione), soprattutto in situazioni con classi sbilanciate o costi diversi a seconda del tipo di errore.
		\item \myalert{Interpretabilità} delle predizione (\textit{perché} è stata scelta una certa classe).
		\item \myalert{Robustezza} dei modelli e \myalert{costo computazionale} per l'apprendimento e per l'esecuzione.
		\item \myalert{Deployment} dei modelli e verifica costante del loro corretto funzionamento.
	\end{itemize}

\end{frame}

\end{document}
