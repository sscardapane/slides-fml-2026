
\documentclass[12pt,aspectratio=169,notheorems]{beamer}

% Packages
\usepackage{mathtools}
\usepackage{hyperref}
\usepackage{setspace}
\usepackage{tikz}
\usetikzlibrary{positioning}
\usepackage{ragged2e}
\usepackage{nicematrix}
\usepackage[compatibility=false]{caption}
\usepackage[framemethod=tikz]{mdframed}

% Colored boxes
\usepackage{tcolorbox}
\tcbsetforeverylayer{autoparskip} % Removes unwanted vertical space

% Theme
\usetheme{metropolis}
\metroset{subsectionpage=progressbar,sectionpage=none}
\setbeamertemplate{theorems}[numbered]

\usepackage{fontspec}
\defaultfontfeatures{LetterSpace=30}


% Code
\usepackage{listings}
% Custom colors
\definecolor{codegreen}{rgb}{0,0.6,0}
\definecolor{codegray}{rgb}{0.5,0.5,0.5}
\definecolor{codepurple}{rgb}{0.58,0,0.82}
\definecolor{backcolour}{rgb}{0.95,0.95,0.92}
\definecolor{light-gray}{gray}{0.95}
% Python style for highlighting
\newcommand\pythonstyle{\lstset{
		backgroundcolor=\color{white},
		language=Python,
		basicstyle=\ttfamily\footnotesize,
		morekeywords={self},              % Add keywords here
		keywordstyle=\color{blue},
		stringstyle=\footnotesize\color{deepgreen},
		commentstyle=\color{codegreen},
		showstringspaces=false,
		frame=tb, 
		numbers=left,
		numbersep=5pt,
		numberstyle=\tiny\color{gray},
		columns=fullflexible,
		numbersep=5pt,
		aboveskip=1em,
		showtabs=false,
		tabsize=2,
		gobble=3
}}

% Python environment
\lstnewenvironment{python}[1][]
{
	\pythonstyle
	\lstset{#1}
}
{}
% Python for inline
\newcommand\pythoninline[1]{{\pythonstyle\lstinline!#1!}}

% Tables and justification
\newcommand{\ra}[1]{\renewcommand{\arraystretch}{#1}}
\justifying

% Subfigures
\setbeamertemplate{caption}[numbered]
\usepackage[caption=false,font=footnotesize]{subfig}

% Colors
\setbeamercolor{uppercolgreen}{fg=white,bg=green!35}
\setbeamercolor{lowercolgreen}{fg=black,bg=green!10}
\setbeamercolor{uppercolred}{fg=white,bg=red!35}
\setbeamercolor{lowercolred}{fg=black,bg=red!10}
\setbeamercolor{uppercolblue}{fg=white,bg=blue!35}
\setbeamercolor{lowercolblue}{fg=black,bg=blue!10}
\definecolor{darkred}{rgb}{0.55, 0.0, 0.0}
\definecolor{darkpastelblue}{rgb}{0.47, 0.62, 0.8}
\definecolor{darkcerulean}{rgb}{0.03, 0.27, 0.49}
\definecolor{darkgreen}{rgb}{0.0, 0.55, 0.0}
\definecolor{deepblue}{rgb}{0,0,0.5}
\definecolor{deepred}{rgb}{0.6,0,0}
\definecolor{deepgreen}{rgb}{0,0.5,0}

% Matrix size
\usepackage{stackengine} 
\stackMath
\def\sss{\scriptstyle \color{gray}}
\setstackgap{L}{12pt}
\def\stacktype{L}

% Mathematical commands
\newcommand{\vc}[1]{\boldsymbol{\mathbf{#1}}}
\newcommand{\pr}[1]{{#1\,}'}
\newcommand{\idx}[2]{{\color{darkpastelblue}[}#1{\color{darkpastelblue}]_{#2}}}
\newcommand{\shape}[2]{\stackunder{#1}{\sss #2}}
\DeclareMathOperator*{\argmax}{arg\,max}
\DeclareMathOperator*{\argmin}{arg\,min}

% Style commands
\newcommand{\myalert}[1]{{\color{darkred}\textbf{#1}}}

% White frame (for images)
\newenvironment{whiteframe}[1]{
	\setbeamercolor{background canvas}{bg=white}
	\begin{frame}{#1}
	}{
	\end{frame}
}

\setbeamerfont{frametitle}{size=\normalsize}
\setbeamertemplate{frametitle}[default][right]
{	
}
%\setbeamercolor{frametitle}{fg=white,bg=black!75}

% No headline frame
\makeatletter
\newenvironment{noheadline}{
	\setbeamertemplate{headline}{}
	\addtobeamertemplate{frametitle}{\vspace*{-3.5\baselineskip}}{}
}{}
\makeatother

% Lists
\setbeamertemplate{itemize items}[triangle]

% Footnotes
\setbeamertemplate{footnote}%
{%
	\parindent 1.5em\noindent%
	\justifying\setstretch{0.7}%
	\hbox to 1em{\hfil\insertfootnotemark}\begin{scriptsize}\insertfootnotetext\end{scriptsize}\par%
}

% Footnote without number
\newcommand\blfootnote[1]{%
	\begingroup
	\renewcommand\thefootnote{}\footnote[frame]{\noindent#1}%
	\addtocounter{footnote}{-1}%
	\endgroup
}

% Theorems
\newcommand*{\theorembreak}{\usebeamertemplate{theorem end}\framebreak\usebeamertemplate{theorem begin}}
\newcounter{theo}[section]\setcounter{theo}{0}
\renewcommand{\thetheo}{\arabic{section}.\arabic{theo}}

\newenvironment{theo}[2][]{%
	\refstepcounter{theo}%
	\ifstrempty{#1}%
	{\mdfsetup{%
			frametitle={%
				\tikz[baseline=(current bounding box.east),outer sep=0pt]
				\node[anchor=east,rectangle,fill=red!10]
				{\strut Theorem~\thetheo};}}
	}%
	{\mdfsetup{%
			frametitle={%
				\tikz[baseline=(current bounding box.east),outer sep=0pt]
				\node[anchor=east,rectangle,fill=red!10]
				{\strut Theorem~\thetheo:~#1};}}%
	}%
	\mdfsetup{innertopmargin=5pt,linecolor=red!10,%
		linewidth=2pt,topline=true,%
		frametitleaboveskip=\dimexpr-\ht\strutbox\relax
	}
	\begin{mdframed}[]\relax%
		\label{#2}}{\end{mdframed}}

\usepackage{booktabs}

% Custom block for ML callouts
\newtcolorbox{mlblock}[1]{
  colback=blue!5,
  colframe=blue!35,
  fonttitle=\bfseries,
  title=#1,
  arc=4pt,
  outer arc=4pt,
  left=5pt,
  right=5pt,
  top=5pt,
  bottom=5pt,
  boxrule=1pt
}

\begin{document}

{\usebackgroundtemplate{\tikz\node[opacity=0.4,inner sep=0]{\includegraphics[width=\paperwidth,height=\paperheight]{images/header}};}%
\begin{frame}[plain]
    \vspace{0.5cm}
    \title{\large \begin{spacing}{1.0}Fondamenti di Machine Learning\end{spacing}\vspace{0.25em}
        \normalsize \begin{spacing}{1.0}\textbf{Laurea Triennale in Ingegneria delle Comunicazioni}\end{spacing}\vspace{0.5em}}
    \subtitle{\Large \textbf{E1 - Python (introduzione)}}
    \date{
        {\includegraphics[scale=0.8]{images/Uniroma1}}
    }
    \author{
        \setlength{\tabcolsep}{2pt}
        \begin{tabular}{rl}
            \textbf{Docente}: & S. Scardapane \\
        \end{tabular}
    }\titlepage
\end{frame}
}

% --- SECTION 0: INTRODUZIONE ---
\section{Introduzione a Python}

\begin{frame}{Obiettivi della sezione}
    \begin{itemize}
        \item Cosa è Python.
        \item Comprendere come viene eseguito il codice (REPL, Script, Notebook).
        \item Saper organizzare un progetto minimo con moduli e import.
    \end{itemize}
\end{frame}

\begin{frame}{Cos'è Python?}
    \begin{itemize}
        \item Linguaggio interpretato ad alto livello, con tipizzazione dinamica.
        \item \textbf{Filosofia}: "Readability counts" (\href{https://peps.python.org/pep-0020/}{The Zen of Python}).
        \item Vasto ecosistema di librerie scientifiche.
    \end{itemize}
    \begin{mlblock}{Perché serve in ML?}
        Velocità di prototipazione + supporto di librerie "heavy-lifting" scritte in C++/CUDA (come PyTorch o NumPy) che garantiscono performance elevate.
    \end{mlblock}
\end{frame}


\begin{frame}{Compilato vs Interpretato}
    \begin{itemize}
        \item \textbf{Linguaggi Compilati} (es. C, C++): Il codice viene tradotto interamente in linguaggio macchina prima dell'esecuzione.
        \item \textbf{Linguaggi Interpretati} (es. Python): Il codice viene letto ed eseguito riga per riga da un \textbf{interprete}.
    \end{itemize}
    \vfill
    \textbf{Vantaggi di Python}:
    \begin{itemize}
        \item Portabilità (gira ovunque ci sia l'interprete).
        \item Sviluppo rapido (no tempi di compilazione).
        \item Facilità di debugging.
    \end{itemize}
\end{frame}

\begin{frame}{Come gira il codice}
    \begin{itemize}
        \item \textbf{REPL (Interattivo)}: Terminale per test rapidi. Input diretto, output immediato.
        \item \textbf{Script (.py)}: Codice salvato in file di testo ed eseguito dall'interprete. Fondamentale per pipeline e produzione.
        \item \textbf{Notebook (.ipynb)}: Celle interattive con codice, testo e grafici. Standard per esperimenti e analisi dati.
    \end{itemize}
\end{frame}

\begin{frame}[fragile]{Esempio: Python REPL}
    Il REPL permette di eseguire codice riga per riga:
    \begin{python}[numbers=none]
        >>> a = 10
        >>> b = 20
        >>> print(a + b)
        30
        >>> type(a)
        <class 'int'>
    \end{python}
    \vfill
    \begin{itemize}
        \item Utile per testare piccole espressioni o ispezionare oggetti.
        \item Si avvia semplicemente digitando \pythoninline{python} nel terminale.
    \end{itemize}
\end{frame}

\begin{frame}{Cosa sono i Moduli?}
    In Python, un \textbf{modulo} è semplicemente un file con estensione \pythoninline{.py} che contiene codice.
    \begin{itemize}
        \item Permettono di organizzare il codice in modo logico e riutilizzabile.
        \item Possiamo usare funzioni definite in altri file tramite il comando \pythoninline{import}.
        \item Es: \pythoninline{import math} ci permette di usare \pythoninline{math.sqrt(4)}.
    \end{itemize}
    \begin{mlblock}{Perché serve in ML?}
        Tutte le principali librerie (NumPy, PyTorch, Scikit-Learn) sono organizzate in moduli e pacchetti che dobbiamo importare per usarne le funzionalità.
    \end{mlblock}
\end{frame}

\begin{frame}[fragile]{Esempio minimo: Moduli e Main}
    \begin{python}
        # utils.py
        def greet(name: str) -> str:
        return f"Ciao, {name}!"

        # main.py
        from utils import greet

        def main():
        print(greet("Python"))

        if __name__ == "__main__":
        main()
    \end{python}
\end{frame}

\begin{frame}{Ambienti e Dipendenze}
    \begin{itemize}
        \item \textbf{Virtual Environments} (\pythoninline{venv}, \pythoninline{conda}): Isolano le librerie di un progetto per evitare conflitti tra versioni diverse.
        \item \textbf{Gestione pacchetti}: Uso di \pythoninline{pip} / \pythoninline{uv} e file di configurazione (es: \pythoninline{requirements.txt}).
    \end{itemize}
    \begin{mlblock}{Perché serve in ML?}
        La riproducibilità è chiave: un esperimento deve poter girare mesi dopo con le stesse identiche versioni delle librerie.
    \end{mlblock}
\end{frame}



\begin{frame}{Errori comuni}
    \begin{itemize}
        \item \myalert{Riproducibilità}: Nei notebook, le celle eseguite fuori ordine possono creare bug difficili da replicare.
        \item \myalert{Working directory}: Lanciare script da cartelle diverse può rompere i path di caricamento dei dati.
        \item \myalert{Install vs Import}: Confondere l'installazione (\pythoninline{pip install}) con l'uso (\pythoninline{import}).
    \end{itemize}
\end{frame}

\begin{frame}{Mini-esercizio: Hello Modules}
    Prova a replicare la struttura vista prima:
    \begin{enumerate}
        \item Crea un file \pythoninline{calc.py} con una funzione \pythoninline{add(a, b)}.
        \item Crea un file \pythoninline{run.py}, importa \pythoninline{add} e stampa il risultato di un'operazione.
        \item Prova a eseguire la stessa funzione definendola direttamente nel REPL (terminale Python).
    \end{enumerate}
\end{frame}

% --- SECTION 1: MENTAL MODEL ---
\section{Mental Model di Python}

\begin{frame}{Obiettivi della sezione}
    \begin{itemize}
        \item Comprendere la distinzione tra \textbf{nomi} e \textbf{valori}.
        \item Capire la differenza tra tipi \textbf{mutabili} e \textbf{immutabili}.
        \item Imparare a gestire l'\textbf{aliasing} e le copie.
    \end{itemize}
\end{frame}

\begin{frame}[fragile]{Nomi vs Valori}
    In Python, le variabili sono \textit{etichette} (nomi) applicate a oggetti (valori) in memoria.
    \begin{python}
        a = [1, 2, 3] # 'a' punta a una lista
        b = a          # 'b' punta allo STESSO oggetto
        b.append(4)
        print(a)       # Output: [1, 2, 3, 4] (Aliasing!)
    \end{python}
    \begin{mlblock}{Perché serve in ML?}
        Molte librerie (come NumPy o PyTorch) restituiscono "viste" dei dati invece di copie per efficienza. Modificare una vista modifica l'originale!
    \end{mlblock}
\end{frame}

\begin{frame}[fragile]{Oggetti e tipi}
    \begin{itemize}
        \item \textbf{Strong typing}: I tipi vanno dichiarati ovunque e sono \textit{vincolanti}.
        \item \textbf{Weak typing} (Python): Non ci sono controlli sui tipi di dato passati alle varie funzioni, si possono creare situazioni molto particolari (es., \textit{duck typing}).
    \end{itemize}
    \textbf{Type Hints}: Suggerimenti non vincolanti a runtime.
    \begin{python}
        def process_data(value: float, name: str) -> str:
        return f"{name}: {value}"
    \end{python}
\end{frame}

\begin{frame}[fragile]{Mutabilità vs Immutabilità}
    \begin{itemize}
        \item \textbf{Immutabili}: int, float, str, tuple, bool. (Non possono cambiare "in-place")
        \item \textbf{Mutabili}: list, dict, set. (Possono cambiare stato)
    \end{itemize}
    \begin{python}
        s = "ciao"
        # s[0] = "C" # Errore!

        l = [1, 2]
        l[0] = 10    # OK!
    \end{python}
\end{frame}

\begin{frame}[fragile]{Assegnamento vs Copia}
    Per evitare l'aliasing, dobbiamo creare una copia esplicita:
    \begin{python}
        a = [1, 2, 3]
        b = a.copy() # Copia shallow
        b.append(4)
        print(a) # [1, 2, 3] - a è preservato!
    \end{python}
    \vfill
    \textbf{Identity vs Equality}:
    \begin{itemize}
        \item \pythoninline{a == b}: Controlla se i \textit{valori} sono uguali.
        \item \pythoninline{a is b}: Controlla se sono lo \textit{stesso oggetto} (stesso indirizzo).
    \end{itemize}
\end{frame}

% --- SECTION 2: TIPI BASE E CONTROLLO ---
\section{Tipi Base e Controllo di Flusso}

\begin{frame}{Obiettivi della sezione}
    \begin{itemize}
        \item Ripassare i tipi numerici e booleani.
        \item Padroneggiare i costrutti \pythoninline{if}, \pythoninline{for} e \pythoninline{while}.
        \item Comprendere il concetto di \textit{truthy/falsy}.
    \end{itemize}
\end{frame}

\begin{frame}[fragile]{Variabili e Tipi}
    \begin{python}
        a = 5         # Integer
        b = 4.2       # Float
        c = 'cat'     # String
        d = (4 == 3)  # Boolean (False)

        print(type(a)) # <class 'int'>
    \end{python}
    \begin{mlblock}{Perché serve in ML?}
        Diversi tipi di dato richiedono tipi di variabili diversi. Inoltre, la precisione numerica (float32 vs float64) è fondamentale per le performance dei modelli.
    \end{mlblock}
\end{frame}


\begin{frame}[fragile]{Visualizzare i Dati: f-strings}
    Python offre diversi modi per stampare variabili:
    \begin{itemize}
        \item \textbf{Standard}: \pythoninline{print("Il valore è", x)}
        \item \textbf{f-strings} (Python 3.6+): \pythoninline{print(f"Il valore è {x}")}
    \end{itemize}
    \begin{python}
        acc = 0.95
        epoch = 10

        # f-strings permettono espressioni e formattazione
        print(f"Epoch {epoch}: accuracy = {acc:.2%}")
        # Output: Epoch 10: accuracy = 95.00%
    \end{python}

\end{frame}

\begin{frame}[fragile]{Controllo di Flusso: If}
    \begin{python}
        x = 10
        if x > 5:
        print("Grande")
        elif x == 5:
        print("Medio")
        else:
        print("Piccolo")
    \end{python}
    \textbf{Truthy/Falsy}: Valori "vuoti" (\pythoninline{0}, \pythoninline{[]}, \pythoninline{""}, \pythoninline{None}) sono valutati come \pythoninline{False}.
\end{frame}

\begin{frame}[fragile]{Iterazione: For e Range}
    \begin{python}
        # range(start, stop, step)
        for i in range(0, 5, 2):
        print(i) # 0, 2, 4

        # Iterare su elementi
        names = ["ML", "AI", "DL"]
        for name in names:
        if name == "AI":
        continue # Salta
        print(name)
    \end{python}
\end{frame}

\begin{frame}{Mini-esercizio: Controllo Flusso}
    Scrivi un ciclo che stampa tutti i numeri da 1 a 20, ma:
    \begin{itemize}
        \item Se il numero è divisibile per 3, stampa "Machine".
        \item Se è divisibile per 5, stampa "Learning".
        \item Se è divisibile per entrambi, stampa "Machine Learning".
    \end{itemize}
\end{frame}

\begin{frame}{Errori comuni}
    \begin{itemize}
        \item \myalert{Indentazione}: Python usa gli spazi, non le graffe!
        \item \myalert{Off-by-one}: \pythoninline{range(5)} arriva a 4, non 5.
        \item \myalert{Modifica durante l'iterazione}: Mai rimuovere elementi da una lista mentre ci stai iterando sopra.
    \end{itemize}
\end{frame}

% --- SECTION 3: SEQUENZE E COLLEZIONI ---
\section{Sequenze e Collezioni}

\begin{frame}{Obiettivi della sezione}
    \begin{itemize}
        \item Padroneggiare le liste (indexing e slicing).
        \item Usare list comprehensions per codice sintetico.
        \item Gestire dizionari per configurazioni e parametri.
        \item Sfruttare \pythoninline{enumerate} e \pythoninline{zip}.
    \end{itemize}
\end{frame}

\begin{frame}[fragile]{Liste: Indexing e Slicing}
    \begin{python}
        l = [10, 20, 30, 40, 50]
        print(l[0])    # 10
        print(l[-1])   # 50 (ultimo)

        # Slicing: [start:stop:step]
        print(l[1:3])  # [20, 30]
        print(l[::-1]) # [50, 40, 30, 20, 10] (reverse)
    \end{python}
    \begin{mlblock}{Perché serve in ML?}
        Lo slicing è il modo standard per fare "mini-batching" o selezionare porzioni di un dataset.
    \end{mlblock}
\end{frame}

\begin{frame}[fragile]{List Comprehensions}
    Modo rapido per creare liste:

    \pythoninline{[expression for item in iterable if condition]}

    \begin{python}
        data = [1, 2, 3, 4, 5, 6]
        # Seleziona solo i pari e fanne il quadrato
        squares = [x**2 for x in data if x % 2 == 0]
        # squares: [4, 16, 36]
    \end{python}
\end{frame}


\begin{frame}[fragile]{Tuple vs Liste}
    Le \textbf{tuple} sono simili alle liste, ma sono \textbf{immutabili}.
    \begin{itemize}
        \item \textbf{Sintassi}: \pythoninline{t = (1, 2, 3)} (parentesi tonde).
        \item Una volta creata, non puoi aggiungere o modificare elementi.
    \end{itemize}
    \begin{python}
        l = [1, 2]
        l[0] = 10    # OK

        t = (1, 2)
        # t[0] = 10  # Errore!
    \end{python}
    \begin{mlblock}{Perché serve in ML?}
        Le tuple vengono spesso usate per rappresentare le \textbf{dimensioni} (shape) di un'immagine o di un dataset: \pythoninline{shape = (224, 224, 3)}.
    \end{mlblock}
\end{frame}

\begin{frame}[fragile]{Attenzione alle parentesi}
    In Python, l'uso di più parentesi indica spesso \textbf{strutture annidate}.
    \begin{python}
        # Una lista di tuple
        points = [(0, 0), (1, 1), (2, 2)]

        # Passare una tupla come singolo argomento
        # random.sample vuole (popolazione, k)
        import random
        res = random.sample((10, 20, 30), 2)
    \end{python}
    \vfill
    \begin{itemize}
        \item \pythoninline{func((1, 2))}: Stai passando \textbf{un solo} argomento (una tupla).
        \item \pythoninline{func(1, 2)}: Stai passando \textbf{due} argomenti distinti.
    \end{itemize}
\end{frame}

\begin{frame}[fragile]{Dizionari}
    Coppie chiave-valore. Utili come contenitori di iper-parametri.
    \begin{python}
        config = {
        'lr': 0.01,
        'epochs': 10,
        'optimizer': 'Adam'
        }
        print(config['lr'])
        config['batch_size'] = 32
    \end{python}
    \textbf{Iterazione}:
    \begin{python}
        for k, v in config.items():
        print(f"{k} -> {v}")
    \end{python}
\end{frame}

\begin{frame}[fragile]{Metodi dei Dizionari}
    Per accedere ai contenuti di un dizionario possiamo usare:
    \begin{itemize}
        \item \pythoninline{.keys()}: Restituisce le chiavi.
        \item \pythoninline{.values()}: Restituisce i valori.
        \item \pythoninline{.items()}: Restituisce coppie (chiave, valore).
    \end{itemize}
    \begin{python}
        d = {"lr": 0.01, "optimizer": "Adam"}

        print(d.keys())   # dict_keys(['lr', 'optimizer'])
        print(d.values()) # dict_values([0.01, 'Adam'])
        print(d.items())  # dict_items([('lr', 0.01), ...])
    \end{python}
\end{frame}


\begin{frame}[fragile]{Enumerate e Zip}
    \begin{itemize}
        \item \textbf{Enumerate}: Indice + Valore.
        \item \textbf{Zip}: Mette insieme più sequenze.
    \end{itemize}
    \begin{python}
        inputs = [0.1, 0.2, 0.3]
        labels = [0, 1, 0]

        for i, (x, y) in enumerate(zip(inputs, labels)):
        print(f"Esempio {i}: Input {x}, Label {y}")
    \end{python}
\end{frame}

\begin{frame}{Mini-esercizio: Collezioni}
    Data una lista di stringhe \pythoninline{["train", "test", "val"]}, usa una list comprehension per creare una lista di tuple \pythoninline{(indice, stringa_maiuscola)}.
    \\ \textit{Risultato atteso: [(0, 'TRAIN'), (1, 'TEST'), ...]}
\end{frame}

\begin{frame}{Errori comuni}
    \begin{itemize}
        \item \myalert{KeyError}: Cercare una chiave inesistente in un dict (usa \pythoninline{.get()}).
        \item \myalert{Performance}: Cercare in una lista è $O(n)$, in un set/dict è $O(1)$.
        \item \myalert{Mutabilità}: Ricordare che una lista come chiave di un dict \textbf{non è ammessa} (usa le tuple!).
    \end{itemize}
\end{frame}

% --- SECTION 4: FUNZIONI ---
\section{Funzioni}

\begin{frame}{Obiettivi della sezione}
    \begin{itemize}
        \item Definire funzioni con parametri opzionali.
        \item Distinguere tra parametri posizionali e keyword.
        \item Capire il ritorno di valori multipli.
    \end{itemize}
\end{frame}

\begin{frame}[fragile]{Definizione e Default Args}
    \begin{python}
        def train_model(data, lr=0.01, model_type="CNN"):
        """Docstring: addestra un modello."""
        print(f"Training {model_type} with lr={lr}")
        return "Success", 0.95 # Ritorna una tupla
    \end{python}
    \textbf{Chiamate flessibili}:
    \begin{python}
        res, acc = train_model(my_data, model_type="Transformer")
    \end{python}
\end{frame}

\begin{frame}[fragile]{Funzioni come Oggetti e Lambda}
    In Python le funzioni sono "first-class citizens".
    \begin{python}
        def double(x): return x * 2
        f = double
        print(f(5)) # 10

        # Lambda (anonime)
        f_add = lambda x, y: x + y
        print(f_add(2, 3)) # 5
    \end{python}
\end{frame}

\begin{frame}{Errori comuni}
    \begin{itemize}
        \item \myalert{Default mutabili}: Mai usare \pythoninline{def f(l=[])}. La lista \pythoninline{l} viene creata una volta sola e riutilizzata tra le chiamate! Usa \pythoninline{None} come default.
        \item \myalert{Scope}: Variabili definite dentro la funzione non sono visibili fuori.
        \item \myalert{Shadowing}: Non chiamare variabili con i nomi di funzioni builtin (es. \pythoninline{list = [1, 2]}).
    \end{itemize}
\end{frame}

% --- SECTION 5: MODULI ---
\section{Moduli e Standard Library}

\begin{frame}[fragile]{Importare Codice}
    \begin{python}
        import math
        print(math.sqrt(16))

        from random import shuffle, seed
        seed(42) # Riproducibilità!

        import os
        print(os.getcwd()) # Directory corrente
    \end{python}
    \begin{mlblock}{Perché serve in ML?}
        Il modulo \pythoninline{random} è vitale per lo split dei dataset e l'inizializzazione dei pesi. La riproducibilità (\pythoninline{seed}) è fondamentale.
    \end{mlblock}
\end{frame}

% --- SECTION 7: DEBUGGING ---
\section{Debugging e Errori}

\begin{frame}[fragile]{Leggere un Traceback}
    Non aver paura degli errori! L'ultima riga dice \textbf{cosa}, le righe sopra dicono \textbf{dove}.
    \begin{python}
        l = [1, 2]
        # l[5] -> IndexError: list index out of range

        # d = {"a":1}
        # d["b"] -> KeyError: 'b'
    \end{python}
    \textbf{Asserzioni}: Utili per sanity check nei training loop.
    \begin{python}
        assert x > 0, "Il learning rate deve essere positivo!"
    \end{python}
\end{frame}

\begin{frame}[fragile]{Gestione Errori}
    \begin{python}
        try:
        val = int(input("Inserisci un numero: "))
        except ValueError:
        print("Non era un numero!")
    \end{python}
\end{frame}

\end{document}
