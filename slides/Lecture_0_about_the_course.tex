
\documentclass[12pt,aspectratio=169,notheorems]{beamer}

% Packages
\usepackage{mathtools}
\usepackage{hyperref}
\usepackage{setspace}
\usepackage{tikz}
\usetikzlibrary{positioning}
\usepackage{ragged2e}
\usepackage{nicematrix}
\usepackage[compatibility=false]{caption}
\usepackage[framemethod=tikz]{mdframed}

% Colored boxes
\usepackage{tcolorbox}
\tcbsetforeverylayer{autoparskip} % Removes unwanted vertical space

% Theme
\usetheme{metropolis}
\metroset{subsectionpage=progressbar,sectionpage=none}
\setbeamertemplate{theorems}[numbered]

\usepackage{fontspec}
\defaultfontfeatures{LetterSpace=30}


% Code
\usepackage{listings}
% Custom colors
\definecolor{codegreen}{rgb}{0,0.6,0}
\definecolor{codegray}{rgb}{0.5,0.5,0.5}
\definecolor{codepurple}{rgb}{0.58,0,0.82}
\definecolor{backcolour}{rgb}{0.95,0.95,0.92}
\definecolor{light-gray}{gray}{0.95}
% Python style for highlighting
\newcommand\pythonstyle{\lstset{
		backgroundcolor=\color{white},
		language=Python,
		basicstyle=\ttfamily\footnotesize,
		morekeywords={self},              % Add keywords here
		keywordstyle=\color{blue},
		stringstyle=\footnotesize\color{deepgreen},
		commentstyle=\color{codegreen},
		showstringspaces=false,
		frame=tb, 
		numbers=left,
		numbersep=5pt,
		numberstyle=\tiny\color{gray},
		columns=fullflexible,
		numbersep=5pt,
		aboveskip=1em,
		showtabs=false,
		tabsize=2,
		gobble=3
}}

% Python environment
\lstnewenvironment{python}[1][]
{
	\pythonstyle
	\lstset{#1}
}
{}
% Python for inline
\newcommand\pythoninline[1]{{\pythonstyle\lstinline!#1!}}

% Tables and justification
\newcommand{\ra}[1]{\renewcommand{\arraystretch}{#1}}
\justifying

% Subfigures
\setbeamertemplate{caption}[numbered]
\usepackage[caption=false,font=footnotesize]{subfig}

% Colors
\setbeamercolor{uppercolgreen}{fg=white,bg=green!35}
\setbeamercolor{lowercolgreen}{fg=black,bg=green!10}
\setbeamercolor{uppercolred}{fg=white,bg=red!35}
\setbeamercolor{lowercolred}{fg=black,bg=red!10}
\setbeamercolor{uppercolblue}{fg=white,bg=blue!35}
\setbeamercolor{lowercolblue}{fg=black,bg=blue!10}
\definecolor{darkred}{rgb}{0.55, 0.0, 0.0}
\definecolor{darkpastelblue}{rgb}{0.47, 0.62, 0.8}
\definecolor{darkcerulean}{rgb}{0.03, 0.27, 0.49}
\definecolor{darkgreen}{rgb}{0.0, 0.55, 0.0}
\definecolor{deepblue}{rgb}{0,0,0.5}
\definecolor{deepred}{rgb}{0.6,0,0}
\definecolor{deepgreen}{rgb}{0,0.5,0}

% Matrix size
\usepackage{stackengine} 
\stackMath
\def\sss{\scriptstyle \color{gray}}
\setstackgap{L}{12pt}
\def\stacktype{L}

% Mathematical commands
\newcommand{\vc}[1]{\boldsymbol{\mathbf{#1}}}
\newcommand{\pr}[1]{{#1\,}'}
\newcommand{\idx}[2]{{\color{darkpastelblue}[}#1{\color{darkpastelblue}]_{#2}}}
\newcommand{\shape}[2]{\stackunder{#1}{\sss #2}}
\DeclareMathOperator*{\argmax}{arg\,max}
\DeclareMathOperator*{\argmin}{arg\,min}

% Style commands
\newcommand{\myalert}[1]{{\color{darkred}\textbf{#1}}}

% White frame (for images)
\newenvironment{whiteframe}[1]{
	\setbeamercolor{background canvas}{bg=white}
	\begin{frame}{#1}
	}{
	\end{frame}
}

\setbeamerfont{frametitle}{size=\normalsize}
\setbeamertemplate{frametitle}[default][right]
{	
}
%\setbeamercolor{frametitle}{fg=white,bg=black!75}

% No headline frame
\makeatletter
\newenvironment{noheadline}{
	\setbeamertemplate{headline}{}
	\addtobeamertemplate{frametitle}{\vspace*{-3.5\baselineskip}}{}
}{}
\makeatother

% Lists
\setbeamertemplate{itemize items}[triangle]

% Footnotes
\setbeamertemplate{footnote}%
{%
	\parindent 1.5em\noindent%
	\justifying\setstretch{0.7}%
	\hbox to 1em{\hfil\insertfootnotemark}\begin{scriptsize}\insertfootnotetext\end{scriptsize}\par%
}

% Footnote without number
\newcommand\blfootnote[1]{%
	\begingroup
	\renewcommand\thefootnote{}\footnote[frame]{\noindent#1}%
	\addtocounter{footnote}{-1}%
	\endgroup
}

% Theorems
\newcommand*{\theorembreak}{\usebeamertemplate{theorem end}\framebreak\usebeamertemplate{theorem begin}}
\newcounter{theo}[section]\setcounter{theo}{0}
\renewcommand{\thetheo}{\arabic{section}.\arabic{theo}}

\newenvironment{theo}[2][]{%
	\refstepcounter{theo}%
	\ifstrempty{#1}%
	{\mdfsetup{%
			frametitle={%
				\tikz[baseline=(current bounding box.east),outer sep=0pt]
				\node[anchor=east,rectangle,fill=red!10]
				{\strut Theorem~\thetheo};}}
	}%
	{\mdfsetup{%
			frametitle={%
				\tikz[baseline=(current bounding box.east),outer sep=0pt]
				\node[anchor=east,rectangle,fill=red!10]
				{\strut Theorem~\thetheo:~#1};}}%
	}%
	\mdfsetup{innertopmargin=5pt,linecolor=red!10,%
		linewidth=2pt,topline=true,%
		frametitleaboveskip=\dimexpr-\ht\strutbox\relax
	}
	\begin{mdframed}[]\relax%
		\label{#2}}{\end{mdframed}}


\begin{document}

{\usebackgroundtemplate{\tikz\node[opacity=0.4,inner sep=0]{\includegraphics[width=\paperwidth,height=\paperheight]{images/header}};}%
\begin{frame}[plain]
	\vspace{0.5cm}
	\title{\large \begin{spacing}{1.0}Fondamenti di Machine Learning\end{spacing}\vspace{0.25em}
		\normalsize \begin{spacing}{1.0}\textbf{Laurea Triennale in Ingegneria delle Comunicazioni}\end{spacing}\vspace{0.5em}}
	\subtitle{\Large \textbf{0 - Descrizione del corso}}
	\date{
		{\includegraphics[scale=0.8]{images/Uniroma1}}
	}
	\author{
		\setlength{\tabcolsep}{2pt}
		\begin{tabular}{rl}
			\textbf{Docente}: & S. Scardapane \\
		\end{tabular}
	}\titlepage
\end{frame}
}

\part{1}

\begin{frame}{Orari ed organizzazione}

	\begin{itemize}
		\item \textbf{Laurea Triennale in Ingegneria delle Comunicazioni}, codice 10600240, terzo anno (secondo semestre), SSD ING-IND/31.
		\item \textbf{Orari}: TBD.
		\item \textbf{Ricevimento}: su appuntamento, online o di persona (Via Eudossiana 18, Dipartimento DIET, primo piano, stanza 102).
	\end{itemize}

	\begin{tcolorbox}
		Pagina web del corso: \newline
		\url{https://www.sscardapane.it/teaching/fml-2026/}.
	\end{tcolorbox}

	\underline{Importante}: registratevi su \textbf{Google Classroom} (dal sito) per rimanere aggiornati su lezioni ed avvisi importanti.

\end{frame}

\begin{frame}{Date esami}

	\begin{enumerate}
		\item \textbf{Sessione III}: \textbf{Giugno 2026}, \textbf{Luglio 2026}.
		\item \textbf{Sessione IV}: \textbf{Settembre 2026}.
		\item \textbf{Sessione II straordinaria}: \textbf{Novembre 2026} (riservata).
	\end{enumerate}

	Seguiranno gli appelli di Gennaio 2027 (I), Febbraio 2027 (II), ed Aprile 2027 (I straordinaria).

\end{frame}

\begin{frame}{Modalità di esame}

	\begin{enumerate}
		\item Un esame scritto, che copre sia gli aspetti teorici (metodologici) che gli aspetti pratici, con un mix di domande aperte e domande a crocette.
	\end{enumerate}

\end{frame}

\begin{frame}{Obiettivi del corso}

	\begin{itemize}
		\item Introduzione ai concetti essenziali di Python: tipi di dato, contenitori (liste, tuple, dizionari), controllo di flusso, funzioni, classi (cenni).
		\item Conoscenza dei concetti base del machine learning: apprendimento supervisionato, bias e varianza, overfitting, ...
		\item Algoritmi di apprendimento supervisionato: modelli lineari, alberi decisionali, ensemble, $k$-NN.
		\item Algoritmi per la riduzione della dimensionalità e per il clustering.
		\item Aspetti di fairness, interpretabilità, e robustezza.
		\item Conoscenza pratica delle principali librerie Python per il machine learning: NumPy, Pandas, scikit-learn, PyTorch (cenni).
	\end{itemize}

\end{frame}

\begin{frame}{Libro di testo}

	Le slide sono autocontenute, per materiale aggiuntivo su specifici argomenti si consiglia di contattare il docente.

	Relativamente alle reti neurali, si consigliano \textit{Understanding Deep Learning} (Prince, 2023), \textit{Dive Into Deep Learning}, \textit{Fundamentals of Deep Learning} (Bishop e Bishop, 2023), o \textit{Alice's adventures in a differentiable wonderland} (2025, del docente, disponibile online).

\end{frame}

\begin{frame}{Python}

	Le esercitazioni pratiche verranno fatte in Python. Per seguirle:

	\begin{itemize}
		\item Per la prima parte del corso è possibile usare Google Colab (\url{https://colab.research.google.com/}), un servizio gratuito per eseguire notebook in Python (accessibile con le credenziali studenti.uniroma1.it).
		\item Per la parte finale, è possibile installare un ambiente completo sul proprio computer con conda (\url{https://www.anaconda.com/}), o direttamente installando Python ed i pacchetti richiesti con \texttt{pip} o \texttt{uv}.
	\end{itemize}

	Come editor, consiglio Spyder per una esperienza simile a MATLAB, oppure Visual Studio Code per una esperienza più completa e simile ad un ambiente di produzione (o eventuali cloni, es., Cursor, Antigravity).

\end{frame}

\end{document}
